%%%%%%%%%%%%%%%%%%%%%%%%%%%%%%%%%%%%%%%%%%%%%%%%%%%%%%%%%%%%%%%%%%%%%%%%%%%%%%%%
%2345678901234567890123456789012345678901234567890123456789012345678901234567890
%        1         2         3         4         5         6         7         8
%
%\documentclass[letterpaper, 10 pt, conference]{ieeeconf}  % Comment this line out if you need a4paper
%
\documentclass[a4paper, 10pt, conference]{ieeeconf}      % Use this line for a4 paper
%
%\IEEEoverridecommandlockouts                              % This command is only needed if 
                                                          % you want to use the \thanks command
%
\overrideIEEEmargins                                      % Needed to meet printer requirements.
%
% See the \addtolength command later in the file to balance the column lengths
% on the last page of the document
%
\usepackage{graphicx} % for pdf, bitmapped graphics files
%\usepackage{hyperref}
%\hypersetup{colorlinks,urlcolor=blue,linkcolor=blue}
%
\usepackage{epstopdf}
%\usepackage{mathptmx} % assumes new font selection scheme installed
%\usepackage{times} % assumes new font selection scheme installed
\usepackage{amsmath} % assumes amsmath package installed
\usepackage{amssymb}  % assumes amsmath package installed

\usepackage{graphicx}
%\usepackage{subfig}
\usepackage{caption}
\usepackage{subcaption}
\usepackage{array}
\usepackage[space]{cite}

%\usepackage[hidelinks]{hyperref}
\usepackage[colorlinks, linkcolor = black, citecolor = black, filecolor = black, urlcolor = blue]{hyperref}


\usepackage{tikz}

\newcommand\encircle[1]{%
	\tikz[baseline=(X.base)] 
	\node (X) [draw, shape=circle, inner sep=0.5pt] {#1};}





%\DeclarePairedDelimiter\abs{\lvert}{\rvert}%
%\DeclarePairedDelimiter\norm{\lVert}{\rVert}%
\DeclareMathOperator*{\argmin}{argmin}
\DeclareMathOperator*{\argmax}{argmax}
\DeclareMathOperator*{\sgn}{sgn}
\newcommand{\specialcell}[2][c]{%
	\begin{tabular}[#1]{@{}c@{}}#2\end{tabular}}

\renewcommand{\citedash}{--}


\newcommand{\etal}{~\textit{et al.}}
%
\title{\bf {\LARGE Modern Intelligent Hand Prostheses} \\ 
{\normalsize H$^2$T-Seminar: Humanoid Robotics, WS 17/18}}
\author{Tobias Stocker, Pascal Weiner and Tamim Asfour \\ High Performance Humanoid Technologies \\ Institute for Anthropomatics and Robotics \\ Karlsruhe Institute of Technology \\
\url{http://www.humanoids.kit.edu}}


%
%
\begin{document}
\maketitle
\thispagestyle{empty}
\pagestyle{empty}
%
%%%%%%%%%%%%%%%%%%%%%%%%%%%%%%%%%%%%%%%%%%%%%%%%%%%%%%%%%%%%%%%%%%%%%%%%%%%%%%%%
\begin{abstract}
Hand Prostheses are an important research topic because these prostheses can improve the quality of life of amputees. The main goal in the design and development of prosthetic hands is to reach a high level of acceptance of the prostheses by potential users. Therefore the hand should look anthropomorphic and be rather lightweight while still enabling the user to do many different grasps with enough force and in short time without too much effort of the user. Another important factor is to keep the prosthetic hands as low-cost as possible to make them affordable for a lot of people.\\
In this paper different hand prostheses which were published in the last few years will be described and compared with regard to various properties like weight, size and degrees of freedom as well as kinematics, mechanics and sensor systems. Additionally special features of the different prosthetic hands will be elaborated.
\end{abstract}

%%%%%%%%%%%%%%%%%%%%%%%%%%%%%%%%%%%%%%%%%%%%%%%%%%%%%%%%%%%%%%%%%%%%%%%%%%%%%%%%
\section{Introduction}

\section{Overview}

\subsection{SSSA-MyHand}

The MyHand~\cite{myhand} was developed by the BioRobotics Institute of the SSSA and published in 2016. The goal was to design a dexterous lightweight hand prosthesis as an alternative to clinically available multi-grasp prostheses while using low-cost manufacturing processes and components wherever possible. To reduce complexity the hand carries three identical 8W brushless DC motors, one for the thumb, one for the index finger and one for the other three fingers. The functional components are hold together by a thin plate surrounded by a 3D-printed metallic mainframe and plastic covers for protection. The hand contains a sensory system for automatic grasp control and makes a future integration of a sensory feedback system possible, e.g. touch sensors in the fingertips. The motors are controlled by the master microcontroller which also acquires the EMG singals and communicates with the external world. The master microcontroller gains information about the actual speed and position of the motors from the slave microcontroller.\\
The force exerted at the fingertips is on average 31.4 N for the thumb, 11.7 N for the index finger and between 9.4 N and 14.6 for the other three fingers. The flexion/extension speed is 160 $^\circ$/s for the thumb and 170 $^\circ$/s for the other fingers, while the speed of the thumb while switching from the opposition to the reposition state can reach 250 $^\circ$/s. The time needed to complete a grasp starting from the rest position is 270 ms for a lateral grasp and 370 ms for a cylindrical grasp.


\subsection{AstoHand v.1}

The AstoHand v.1~\cite{astohand} was developed by the Department of Mechanical Engineering of the Diponegoro University. They focused on designing a low-cost anthropomorphic prosthetic hand which uses DC micro metal gear motors and 3D-printed material to make it lightweight. The developed hand has five degrees of freedom and each finger contains two joints and one motor. The five DoFs were chosen because they are sufficient for the most important activities of daily living while reducing the complexity of the mechanical design and manufacturing costs. For the movement of the fingers the motors are connected to the joints via a tendon-spring mechanism. The five DC motors are placed in the palm, an Arduino Nano microcontroller and driver motor are placed in the back cover and two batteries, with voltage of 3.7 V and capacity of 2425 mAh, are placed in the socket of the hand.

\subsection{Bionic Hand}

The Bionic Hand~\cite{bionichand} is a hybrid actuated prosthetic hand with 24 DoFs developed by the Department of Biomedical Engineering of the Bogazici University and the Department of Mechatronic Engineering of the Marmara University. For the design of the prosthesis the human hand is used as reference with all its joints and connections. Neural Networks are used to classify the EMG signals as well as to control the actuators which include brushless DC motors and Shape Memory Alloy (SMA) actuators. The two types of actuators are used to imitate extrinsic and intrinsic muscles and to emulate the antagonistic contraction. Overall there are 13 motors placed in the forearm, four motor pairs for the flexion and extension of the four fingers, two motors for the flexion-extension and abduction-adduction of the wrist and three motors for controlling the thumb. The SMA actuators are placed on the sides of the metacarpals and are used for the abduction-adduction of the fingers. While they are slow and not strong they are good to imitate the rather weak intrinsic muscles because of their low weight and small size.

\subsection{X-Hand}

The X-Hand~\cite{xhand} is an anthropomorphic prosthetic hand with size similar to the human hand designed by the Institute of Rehabilitation and Medical Robotics of the Huazhong University of Science and Technology. The goal was to design a hand prosthesis which fulfills the following three important features. The hand should have anthropomorphic grasp functions to cover the grasping activities of daily living, few actuators to reduce complexity and easier control and a compact structure with integrated hardware for a flexible installation and convenient wearing for the user.\\
The thumb mechanism of the X-Hand is designed independently of the mechanism of the other fingers to enable the thumb to move independently during grasping like in the human hand. Two DC motors are used for the kinematic transmission system of the four fingers to generate anthropomorphic grasping movements and two DC motors are used for the movement of the thumb. The time needed to close the hand from open is 1.2 s. The grasping force measured during contact with a cylinder between thumb and index finger is 12.1 N. This grasping force is not limited due to the hand construction but rather by the motor size.

\subsection{Six-DOF-Hand}

The Six-DOF-Hand~\cite{6dofhand} is a hand prosthesis developed by the Mechanical Engineering Department of the University of Colorado-Denver. One of the major goals of the project was to design a hand which is inexpensive and also open source. The hand is able to do any of the standard grasps (e.g. tip, palmar, lateral, cylindrical) as well as more complex tasks with independent finger movement. To keep the prosthetic hand low-cost a design with six DOFs was chosen, which includes two actuators for the thumb, for performing flexion/extension and rotation, and one actuator each for the other fingers. The metacarpal phalangeal (MCP) and proximal interphalangeal (PIP) joints are coupled through a belt system, while the distal interphalangeal (DIP) joint is fixed on the fingers. The actuators include encoders to allow motor position feedback and testing and the fingershells of the hand contain internal space to leave the possibility of embedding force sensing elements and flex-sensors open.

\subsection{SoftHand Pro-D}

The SoftHand Pro-D~\cite{softhand} is a prosthetic hand developed by the University of Pisa and the Instituto Italiano di Tecnologia whose design evolves from the Pisa/IIT SoftHand~\cite{pisahand}. The hand is highly underactuated by having 19 DOFs while only a single actuator. The hand consists of passive damping components and the desired motion is accomplished with the help of dynamic synergies. It can move in two different synergistic directions of motion to perform pinch grasps as well as power grasps. The proposed approach tries to exploit the frequency content of the EMG signals in an innovative and natural way, that means the speed of the muscle contraction is associated with the speed of the synergy. The single DC motor is located in the dorsal side of the palm and integrated in a support structure. The damping element lies parallel to the motor and is connected with a compression spring. Only the thumb is directly connected to the damper through a tendon.

\subsection{Tact}

The Tact~\cite{tact} hand is an open-source myoelectric prosthetic hand developed by the Department of Mechanical Science \& Engineering and the Department of Aerospace Engineering of the University of Illinois with the goal to design a hand which is affordable for people in developing countries. Therefore tradeoffs between cost, performance, duration, efficiency and manufacturability were evaluated to find a possible way to make the hand low-cost while still maintaining performance similar to commercial prosthetic hand. They use lower-cost motors and don't use a complex tendon-driven design as joint actuation method but rather a simplified method consisting of a DC motor and a spool. As joint coupling method a four-bar linkage is chosen to produce consistent movement and strong forces at the finger tips.

\subsection{UOMPro}

The UOMPro~\cite{uompro} is an artificial prosthetic hand developed by the Department of Mechanical Engineering of the University of Moratuwa. The designed hand has affordable costs and consists of six DOFs including flexion/extension for all five fingers and abduction/adduction of the thumb. Each finger of the hand consists of two joints, one proximal similar to the human metacarpal phalange (MCP) and one distal similar to both the human proximal interphalange (PIP) and distal interphalange (DIP). The DC geared micro motor in each finger can provide fairly good torque (~0.31 Nm) which is further increased for the proximal joint by a worm and wheel gear system. The fingers are designed with a fixed relationships between the joint motions realized by a form of a four-bar linkage. The flexion/extension motion for the thumb is achieved by a motor on the thumb base also using a four-bar mechanism and the adduction/abduction is achieved by a motor located under the thumb base. The electronic components which are responsible for the low-level controlling are placed inside the hand and a serial communication interface is provided to enable communication to a high-level controller which sends either individual finger positions or hand grip pattern commands.

\subsection{MORA Hap-2}

The MORA Hap-2 is a prosthetic hand with self-adaptive fingers developed by the Department of Mechanical Engineering of the University of Moratuwa. All power transmission attachments and the actuators are placed within the prosthesis and the hand is capable of generating five different grasping patterns (cylindrical grasp, hook grasp, lateral pinch, tip pinch and palmar pinch). All fingers except for the thumb have the same mechanism consisting of two four-bar linkages connected at the PIP joint. Three torsion springs are attached at the joints of the finger to limit joint motion of the four-bar mechanism and to carry out the underactuation. The structure of the thumb is similar to the proximal and middle phalange of the normal finger mechanism. The thumb mechanism consists of a single four-bar linkage and two torsional springs and can generate flexion/extension and adduction/abduction with two separate motors. The finger mechanism of the prosthetic hand includes the ability of self-adaption to provide robust grasps for different objects. When the proximal phalange is in contact with the object during the grasping process, the PIP joint continues to rotate compressing the torsion spring inside the first four-bar linkage until the middle phalange gets in contact with the object. Then the DIP joint continues to rotate until the distal phalange gets in contact with the object and the object is grasped properly. This means the joints of the fingers are capable of adopting their angles passively according to the geometry of the object to ensure a robust grasp.

\subsection{Hand of Bennet et al.}

This hand prosthesis was developed by the Department of Mechanical Engineering of the Vanderbilt University. It incorporates four independent actuators in an unique configuration to provide precision grasps as well as whole-hand grasps. The thumb and the index finger which are responsible for precision grasps are both designed with three fully actuated DOFs to ensure good control during theses grasps. The first motor unit provides abduction/adduction of the thumb and the second motor unit provides flexion/extension of the thumb at the carpometacarpal (CMC) joint while the MCP and PIP joints are fused. The third motor unit provides flexion/extension of the index finger at the MCP joint while the PIP and DIP joints are fused. The other three fingers which are responsible for stabilizing during whole-hand grasping are designed in an underactuated configuration where a single actuator controls all six DOFs to ensure good stability. The thumb and the index finger are actuated via bidirectional tendons while the other fingers are actuated via unidirectional tendons.\\
The hand mainly consists of a high-modulus material to form the load-bearing geometric and kinematic structure and a low-modulus structure to provide a softer cover to facilitate grasping of objects. The hand includes an embedded system consisting of a single, four-layer circuit board which is fully contained within the palm of the prosthetic hand. It accepts and executes motion and force commands from a high-level controller via a controller area network (CAN) serial interface and returns processed position and force information.

\subsection{Hand of Zhang et al.}

This prosthetic hand was developed mainly by the Guangdong Provincial Key Laboratory of Robotics and Intelligent Systems in the Chinese Academy of Sciences and consists of five fingers and 15 joints, with each finger being actuated by one embedded DC motor. It was made out that this is the most balanced approach in terms of efficiency, response time and reliability. The design of the index/middle finger and of the ring/little finger are different from each other to make the motion of the fingers more natural, to make more grasp patters possible and to make them adaptive to the object shape. A linkage mechanism was chosen as actuation mode instead of a tendon driving mechanism because it has more stability, can reach higher grasping forces and need less room. A coupling link mechanism was selected over an underactuated mechanism for the design of the fingers except for the thumb. While the underactuated mechanism makes an adaptive grasping mode possible, it was found out to be too complex and too bulky for their prosthetic hand. Since the thumb plays an important role in most grasping operations they tried to enable the thumb to grasp along a cone surface, similar to the human hand. For this the metacarpal of the thumb is placed at 30$^\circ$ deflection to the base joint of the middle finger and with an initial abduction of 30$^\circ$ to the palm.\\
Dedicated torque and position sensors were developed for the prosthetic hand. They are used to guarantee control schemes such as impedance control in autonomous operations. The hand contains 18 proprioceptive and exteroceptive sensors: the base joints contain five Hall effect joint position sensors and five strain gauge joint torque sensors for measuring the grasping force of each finger, the palm contains five current sensors for the five DC motors and each motor in the thumb, index and middle finger contains three incremental encoders.\\
The mechanical parts, electronics and the control system are embedded in the hand for a modular prosthesis which is easily maintainable. The motion control system consists of a small light weight motion control subsystem with high speed computation and several sensory subsystems. The motion control subsystem includes a DSP chip, an A/D converter to sample the sensor signals for torque and position, a CAN module to communicate with the mutual perception system and a serial communication module to connect the motion control system and a PC for debugging and operator training.\\
A sensory feedback system is used to help the amputees to control the force as they intend. The motion control system acquires and grades the force data and sends it to the man-machine interface. The man-machine interface generates constructive parameters based on a typical stimulation strategy to establish the stimulation wave form. The parameters are then transmitted to the stimulator which produces concrete stimulus waveforms. The stimulus is then applied on the amputee's body so he can feel the actual force exerted by the hand prosthesis.

\subsection{Hand of Mio et al.}

This 3D-printed prosthetic hand was developed by the Laboratory of Biomechanics and Applied Robotics of the Pontificia Universidad Cat\'{o}lica del Per\'{u}. 

\newpage

\hspace*{-1cm}
\begin{table}

\begin{tabular}{l|p{2cm}|p{1cm}|p{1.2cm}|p{2.2cm}|p{1.2cm}|p{1.2cm}|p{1.2cm}|p{2cm}}

Name & Developer & Year & Mass(g) & Size(mm) length x width x thickness & Number of joints & Degrees of freedom & Number of \newline actuators & Actuator type\\
\hline
MyHand~\cite{myhand} & SSSA & 2016 & 478 & 200 x 84 x 56 & 10 & 4 & 3 & Brushless DC Motor\\
\hline
AstoHand v.1~\cite{astohand} & Diponegoro University & 2016 & 261 & 180 x 85 x 50 & 10 & 5 & 5 & DC Motor\\
\hline
Bionic Hand~\cite{bionichand} & Atasoy et al. & 2016 & - & - & 24 & 24 & 13 & Brushless DC Motor\\
\hline
X-Hand~\cite{xhand}& Xiong et al. & 2016 & - & human hand size & 16 & - & 4 & DC Motor\\
\hline
Six-DOF-Hand~\cite{6dofhand} & Krausz et al. & 2016 & 584 & 202 x 99 x 61 & 10 & 6 & 6 & DC Motor\\
\hline
SoftHand Pro-D~\cite{softhand} & Piazza et al. & 2016 & - & - & 19 & 19 & 1 & DC Motor\\
\hline
Tact~\cite{tact} & Slade et al. & 2015 & 350 & 200 x 98 x 27 & 11 & 6 & 6 & DC Motor\\
\hline
UOMPro~\cite{uompro} & Nisal et al. & 2017 & 432 & 199 x 88 x - & 10 & 6 & 6 & DC Micro Motor\\
\hline
MORA Hap-2~\cite{morahap2} & Gopura et al. & 2017 & 250 & 95 (fingers)\newline x 83 x 25 & 14 & 11 & 4 & -\\
\hline
-~\cite{bennett} & Bennett at al. & 2015 & 546 & 200 x 89 x - & 12 & 12 & 4 & Brushless DC Servomotor\\
\hline
-~\cite{zhang} & Zhang et al. & 2015 & 420 & 159 x 79 x 21 & 15 & 5 & 5 & DC Motor\\
\hline
-~\cite{mio} & Mio et al. & 2017 & - & - & 10 & 6 & 6 & DC Micro Motor\\

\end{tabular}

\caption{Table with developer, year of publication, physical properties and information about joints and motors}
\label{table:table1}

\end{table}

\vspace{3cm}

\hspace*{-1cm}

\begin{table}

\begin{tabular}{p{2.5cm}|p{1.2cm}|p{1cm}|p{1.5cm}|p{2cm}|p{2cm}|p{1.2cm}|p{1.5cm}|p{2cm}}

Name / \newline Developer & Number of Fingers & Joints per Finger & Actuators integrated & Transmission system & Sensor system & Gripping force & Individual Finger Force & Joint Speed / Closing Time\\
\hline
MyHand & 5 & 2/2 & Yes & Geneva drive & EMG/position/\newline force & - & 31 N/\newline 12 N & 160-250 $^\circ$/s\\
\hline
Asto Hand v.1 & 5 & 2/2 & Yes & tendon spring & EMG & - & - & -\\
\hline
Bionic Hand & 5 & 3/3 & No & tendons & EMG & - & - & -\\
\hline
X-Hand & 5 & 3/3 & Yes & tendons & - & 12.1 N & - & 1.2 s\\
\hline
Six-Dof-Hand & 5 & 2/2 & Yes & gears/belts & EMG & - & 4.12 N & 2.24 rads/s\\
\hline
SoftHand Pro-D & 5 & 3/3 & Yes & tendons & EMG & - & - & -\\
\hline
Tact & 5 & 3/2 & Yes & tendons & EMG & - & 4.21 N & 249.8 $^\circ$/s\\
\hline
UOMPro & 5 & 2/2 & Yes & four-bar\newline linkage & - & - & - & -\\
\hline
MORA Hap-2 & 5 & 2/3 & Yes & four-bar\newline linkage & - & - & - & -\\
\hline
Bennet et al. & 5 & 3/3/2 & Yes & tendons & tendon\newline excursion & - & 25-30 N & -\\
\hline
Zhang et al. & 5 & 3/3 & Yes & linkage mechanism & EMG/torque/ position & - & 4.3-10 N & 68-118 $^\circ$/s /\newline 1 s\\ 
\hline
Mio et al. & 5 & 2/2 & Yes & four-bar linkage/worm drive & position\newline sensors & - & 1.2-2.4 N & 100-180 $^\circ$/s\\

\end{tabular}

\caption{Table with information about finger kinematics, transmission and sensor systems, forces and joint speeds}

\end{table}

\newpage~\newpage

\section{Comparison}

\subsection{Physical properties}

The average weight of a human hand is 400 g, which is around 0.5 $\%$ of the total body weight for women and around 0.6 $\%$ for men. Since hand prostheses are not connected through the skeleton like a human hand, users feel that prosthetic hands with this weight are too heavy and not comfortable. Therefore one goal to increase acceptability of designed hand prostheses is to keep the weight as low as possible. The weight of the presented hands (see Table~\ref{table:table1}) varies from 250 g for the MORA Hap-2 to 584 g for the Six-DOF-Hand. Only for three out of the eight hands the weight is below the 400 g of a human hand, while the weight of the others is probably too high to wear comfortable for a long time. It is important to note that these weights are the values described by the respective research groups and may not always include all components of the complete hand.\\
Hand prosthesis naturally should have the average size of a human hand or even better the hand size of the actual patient. All presented hands are designed to have more or less the size of a human hand, while the length varies from 159 cm to 202 cm, the width from 79 cm to 99 cm and the thickness from 21 to 61 cm.

\newpage~\newpage

\bibliographystyle{./IEEEtran}
%\nocite{*}
\bibliography{./Ausarbeitung}
%
\end{document}
%