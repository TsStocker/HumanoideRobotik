%%%%%%%%%%%%%%%%%%%%%%%%%%%%%%%%%%%%%%%%%%%%%%%%%%%%%%%%%%%%%%%%%%%%%%%%%%%%%%%%
%2345678901234567890123456789012345678901234567890123456789012345678901234567890
%        1         2         3         4         5         6         7         8
%
%\documentclass[letterpaper, 10 pt, conference]{ieeeconf}  % Comment this line out if you need a4paper
%
\documentclass[a4paper, 10pt, conference]{ieeeconf}      % Use this line for a4 paper
%
%\IEEEoverridecommandlockouts                              % This command is only needed if 
                                                          % you want to use the \thanks command
%
\overrideIEEEmargins                                      % Needed to meet printer requirements.
%
% See the \addtolength command later in the file to balance the column lengths
% on the last page of the document
%
\usepackage{graphicx} % for pdf, bitmapped graphics files
%\usepackage{hyperref}
%\hypersetup{colorlinks,urlcolor=blue,linkcolor=blue}
%
\usepackage{epstopdf}
%\usepackage{mathptmx} % assumes new font selection scheme installed
%\usepackage{times} % assumes new font selection scheme installed
\usepackage{amsmath} % assumes amsmath package installed
\usepackage{amssymb}  % assumes amsmath package installed

\usepackage{graphicx}
%\usepackage{subfig}
\usepackage{caption}
\usepackage{subcaption}
\usepackage{array}
\usepackage[space]{cite}

%\usepackage[hidelinks]{hyperref}
\usepackage[colorlinks, linkcolor = black, citecolor = black, filecolor = black, urlcolor = blue]{hyperref}


\usepackage{tikz}

\newcommand\encircle[1]{%
	\tikz[baseline=(X.base)] 
	\node (X) [draw, shape=circle, inner sep=0.5pt] {#1};}





%\DeclarePairedDelimiter\abs{\lvert}{\rvert}%
%\DeclarePairedDelimiter\norm{\lVert}{\rVert}%
\DeclareMathOperator*{\argmin}{argmin}
\DeclareMathOperator*{\argmax}{argmax}
\DeclareMathOperator*{\sgn}{sgn}
\newcommand{\specialcell}[2][c]{%
	\begin{tabular}[#1]{@{}c@{}}#2\end{tabular}}

\renewcommand{\citedash}{--}


\newcommand{\etal}{~\textit{et al.}}
%
\title{\bf {\LARGE Modern Intelligent Hand Prostheses} \\ 
{\normalsize H$^2$T-Seminar: Humanoid Robotics, WS 16/17}}
\author{Tobias Stocker, Pascal Weiner and Tamim Asfour \\ High Performance Humanoid Technologies \\ Institute for Anthropomatics and Robotics \\ Karlsruhe Institute of Technology \\
\url{http://www.humanoids.kit.edu}}


%
%
\begin{document}
\maketitle
\thispagestyle{empty}
\pagestyle{empty}
%
%%%%%%%%%%%%%%%%%%%%%%%%%%%%%%%%%%%%%%%%%%%%%%%%%%%%%%%%%%%%%%%%%%%%%%%%%%%%%%%%
\begin{abstract}
Hand Prostheses.
\end{abstract}

%%%%%%%%%%%%%%%%%%%%%%%%%%%%%%%%%%%%%%%%%%%%%%%%%%%%%%%%%%%%%%%%%%%%%%%%%%%%%%%%
\section{Introduction}

\section{Hand Prostheses}

The MyHand was developed by the BioRobotics Institute of the SSSA and published in 2016. The goal was to design a dexterous lightweight hand prosthesis as an alternative to clinically available multi-grasp prostheses while using low-cost manufacturing processes and components wherever possible. To reduce complexity the hand carries three identical 8W brushless DC motors, one for the thumb, one for the index finger and one for the other three fingers. The functional components are hold together by a thin plate surrounded by a 3D-printed metallic mainframe and plastic covers for protection. The hand contains a sensory system for automatic grasp control and makes a future integration of a sensory feedback system possible, e.g. touch sensors in the fingertips. The motors are controlled by the master microcontroller which also acquires the EMG singals and communicates with the external world. The master microcontroller gains information about the actual speed and position of the motors from the slave microcontroller.\\
The force exerted at the fingertips is on average 31.4 N for the thumb, 11.7 N for the index finger and between 9.4 N and 14.6 for the other three fingers. The flexion/extension speed is 160 $^\circ$/s for the thumb and 170 $^\circ$/s for the other fingers, while the speed of the thumb while switching from the opposition to the reposition state can reach 250 $^\circ$/s. The time needed to complete a grasp starting from the rest position is 270 ms for a lateral grasp and 370 ms for a cylindrical grasp.


\newpage

.

\newpage

\begin{tabular}{l|p{2cm}|p{1cm}|p{1.2cm}|p{2.2cm}|p{1.2cm}|p{1.2cm}|p{1.2cm}|p{2cm}}

Name & Developer & Year & Mass(g) & Size(mm) length x width x thickness & Number of joints & Degrees of freedom & Number of \newline actuators & Actuator type\\
\hline
MyHand & SSSA & 2016 & 478 & 200 x 84 x 56 & 10 & 4 & 3 & Brushless DC Motor\\
\hline
Asto Hand v.1 & Diponegoro University & 2016 & 261 & 180 x 85 x 50 & 10 & 5 & 5 & DC Motor\\
\hline
Bionic Hand & Atasoy et al. & 2016 & - & - & 24 & 24 & 13 & Brushless DC Motor\\
\hline
X-Hand & Xiong et al. & 2016 & - & human hand size & 16 & - & 4 & DC Motor\\
\hline
Six-DOF-Hand & Krausz et al. & 2016 & 584 & 202 x 99 x 61 & 10 & 6 & 6 & DC Motor\\
\hline
SoftHand Pro-D & Piazza et al. & 2016 & - & - & 19 & 19 & 1 & DC Motor\\
\hline
MORA Hap-2 & Gopura et al. & 2017 & 250 & 95 (fingers)\newline x 83 x 25 & 14 & 11 & 4 & -\\ 

\end{tabular}

\vspace{3cm}

\begin{tabular}{l|p{1.2cm}|p{1cm}|p{1.5cm}|p{2cm}|p{2cm}|p{1.2cm}|p{1.5cm}|p{2cm}}

Name & Number of Fingers & Joints per Finger & Actuators integrated & Transmission system & Sensor system & Gripping force & Individual Finger Force & Joint Speed / Closing Time\\
\hline
MyHand & 5 & 1/2 & Yes & Geneva drive & EMG/automatic grasp control & - & ~31N/~12N & 160-250 $^\circ$/s\\
\hline
Asto Hand v.1 & 5 & 2/2 & Yes & tendon spring & EMG & - & - & -\\
\hline
Bionic Hand & 5 & 3/3 & No & tendons & EMG & - & - & -\\
\hline
X-Hand & 5 & 3/3 & Yes & tendons & - & 12.1N & - & 1.2s\\
\hline
Six-Dof-Hand & 5 & 2/2 & Yes & gears/belts & EMG & - & 4.12N & 2.24 $rads/s$\\
\hline
SoftHand Pro-D & 5 & 3/3 & Yes & tendons & EMG & - & - & -\\
\hline
MORA Hap-2 & 5 & 2/3 & Yes & four-bar\newline linkage & - & - & - & -\\


\end{tabular}

\bibliographystyle{./IEEEtran}
\nocite{*}
\bibliography{./Ausarbeitung}
%
\end{document}
%